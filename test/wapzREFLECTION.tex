\documentclass[iltitle]{../frigaardreportF}

\author{David Frigaard}
\title{How was your understanding of cultural and contextual considerations of the work developed through the interactive oral?}
\date{\today}

\begin{document}
\maketitle
Through my exploration of Nawal El Saadawi's \emph{Women at Point Zero} and the class discussion, I have developed personally and culturally, questioning my knowledge and the values of my society.\\

\noindent During the discussion, I recognised how easily my beliefs have been tainted by media bias, and that I should not believe what I see or read so blindly in the future. Western media often portrays Islam as a misogynistic religion, and it is therefore thought of as under-developed by the West. However, in the novel, it is made apparent that it is not Islam, but the culture that surrounds it, that holds the sexist beliefs. Through class discussion, I understand now that Islam's values and that of its culture are often portrayed as one and the same, with the religion becoming a scapegoat for anti-feminist oppression in the Middle East.\\

\noindent In contrast to Western society's opinions, Firdaus decides that being a prostitute is more respectable than being an office worker. According to her, the latter job means that you are going to someone else for work, whereas as a sex worker, people are coming to you: giving you freedom, money, and power---so why not respect? After this realisation and the discussion, I decided that the definition of respect in Western society is extremely vague---how can we decide if a career is respectable or not? Moreover, I now know to not believe in my society's values without evaluating them beforehand.\\ 
\clearpage

\noindent As Firdaus is unceasingly exploited by male characters, Saadawi's feminist values emerge. This made me think of how women in the West are treated. Admittedly, compared to Egypt of the 1950s, contemporary Western society is many times more equal. Nonetheless, it is easy to feel complacent and believe that we have achieved equality, but through reflection, I recognise that Western societies are still far from perfect. For example, women earn less on average than men in many comparable jobs and women parliamentarians are still a rare occurrence.\\

\noindent In the discussion, I was able to examine and clarify my beliefs regarding murder.  In contrast to the novel and my class' conclusions, who argued that Firdaus' killing of her pimp was symbolic and therefore justified, I found that I could not condone such an action.  While I do not believe that cruelty towards someone is right, nor do I believe that it is permissible to correct such a wrong with murder.\\

\flushright Word Count: 400
\end{document}

eauh eake anf a